\chapter{Introduction}

A medida que el mundo se vuelve cada vez más interconectado, la posibilidad de observar y analizar nuestro entorno desde el espacio se ha convertido en una de las herramientas más importantes para enfrentar los desafíos globales. Desde esta perspectiva, son las misiones satelitales, a través de sus avanzados sistemas de sensores, las que permiten la recolección de información sobre la superficie terrestre, proporcionando la visión más detallada y extensa de los procesos de cambio en nuestro planeta. Entre las disciplinas clave que hacen posibles estas funciones, una de las más destacadas es la teledetección, que transforma los datos obtenidos por estos sensores en información crítica para la resolución de problemas ambientales y sociales. Esta disciplina proporciona análisis específicos sobre las transformaciones y características de determinadas áreas geográficas. Con los avances en la  AI (Artificial Intelligence), la capacidad analítica ha alcanzado un nuevo nivel, permitiendo que casi cualquier tipo de imagen satelital sea analizada con una precisión y eficiencia sin precedentes.

La resolución espacial es un parámetro fundamental para el análisis de imágenes de teledetección. Se define como la distancia mínima en el terreno que separa dos objetos independientes que pueden ser distinguidos. Los factores que la determinan incluyen la altitud, la distancia y la calidad de los instrumentos utilizados \autocite{alparone2015remote}. Un factor adicional importante relacionado con la resolución espacial es la Distancia de Muestreo en el Terreno (GSD), que es la porción de la superficie de la Tierra representada por cada uno de los píxeles \autocite{lillesand2015remote}.

Uno de los sensores más utilizados debido a su alta resolución espacial es el Sentinel-2 MSI, operado por la European Space Agency \autocite{Sentinel2_Handbook}. Desde su lanzamiento en junio de 2015, ha proporcionado imágenes multiespectrales de acceso abierto, lo que ha generado un interés significativo en la comunidad científica y en diversas industrias, convirtiéndose en una herramienta de clave para la provisión de datos en diversas aplicaciones, como la monitorización del uso del suelo, la detección de cambios y el análisis de vegetación. El sensor está equipado con 13 bandas espectrales distribuidas en tres resoluciones espaciales: 4 bandas espectrales con una resolución espacial de 10 m, 6 bandas espectrales con 20 m y 3 bandas espectrales con 60 m, diseñadas para la recolección de diversos parámetros topográficos \autocite{lanaras2018super}. Esta diversidad espectral permite realizar estudios detallados sobre una amplia gama de fenómenos terrestres mediante la combinación de bandas específicas, desde la calidad del agua hasta el monitoreo de superficies nevadas.

Otro de los sensores que destaca por su mayor resolución es el National Agriculture Imagery Program (NAIP), que proporciona imágenes aéreas con una resolución de hasta 1 metro. Aunque su cobertura está limitada al territorio continental de los Estados Unidos, ofrece imágenes aéreas con una resolución espacial de hasta 1 metro, lo que lo convierte en una fuente importante para estudios que requieren un nivel de detalle superior. NAIP captura imágenes durante los meses de mayor actividad agrícola, de junio a agosto, y proporciona actualizaciones cada tres años (anteriormente cada cinco antes de 2009). Desde 2011, las imágenes incluyen bandas RGB y NIR de manera consistente, mejorando su calidad y su aplicabilidad en áreas como la planificación urbana y el monitoreo agrícola. Aunque NAIP y Sentinel-2 presentan enfoques y alcances diferentes, la combinación de imágenes de ambos sensores permite obtener una mayor riqueza de datos, lo que resulta clave para estudios que requieren tanto una cobertura extensa como un nivel de detalle excepcional.

Existen diversas técnicas para abordar la limitación de la resolución espacial y espectral en las imágenes satelitales dentro del campo de la teledetección, desde métodos tradicionales de fusión de imágenes hasta enfoques más recientes basados en inteligencia artificial. Entre estos últimos, el deep learning ha demostrado ser especialmente prometedor \autocite{gargiulo2019fast}. Una de las aplicaciones más efectivas de esta técnica es la superresolución, que busca aumentar la resolución espacial de las imágenes generando detalles adicionales a partir de datos existentes. Estas técnicas de machine learning permiten reconstruir imágenes de mayor resolución a partir de versiones de menor calidad, optimizando tanto la precisión espacial como la coherencia espectral.

\section{Motivation}

El desarrollo de técnicas para mejorar la resolución de las imágenes obtenidas por el sensor Sentinel-2 ha sido un desafío importante en la comunidad científica. En este proceso, arquitecturas avanzadas de deep learning como las Generative Adversarial Networks (GANs) y las redes neuronales convolucionales (CNNs) juegan un papel crucial, gestionando eficientemente los datos \autocite{salgueiro2020super}. Lograr un aumento en la resolución espacial, manteniendo la coherencia espectral y optimizando la calidad de la información, abre nuevas fronteras en el análisis de imágenes satelitales.

El acceso a imágenes gratuitas y de alta calidad, como las proporcionadas por Sentinel-2, es un recurso valioso para la investigación científica. Aunque existen otras formas de obtener imágenes de alta resolución, muchas veces estas no cubren áreas tan extensas o no cuentan con la temporalidad necesaria para ciertos estudios. Gracias a su cobertura global y su alta frecuencia de revisita, este sensor ofrece una ventaja significativa en este sentido. Mejorar su resolución permitiría realizar estudios más detallados sin necesidad de recurrir a sensores costosos, haciendo que los datos disponibles tengan un impacto aún mayor.

Además, este trabajo busca aportar al avance en la aplicación de AI en teledetección. Al utilizar arquitecturas innovadoras, esta investigación tiene el potencial de abrir nuevas posibilidades para mejorar la resolución de imágenes satelitales. Este enfoque no solo beneficia a la ciencia actual, sino que también puede inspirar futuras investigaciones, permitiendo la continuación del desarrollo de estas técnicas en nuevos contextos.


\section{Problem Statement}


A pesar de la disponibilidad de modelos avanzados de superresolución \autocite{salgueiro2020super, navarro_sánchez_2020}, el principal desafío radica en cómo garantizar que las imágenes resultantes no solo mejoren la resolución espacial, sino que también mantengan la coherencia espectral, un factor crucial en aplicaciones científicas. Para superar las limitaciones de resolución espacial de los sensores como Sentinel-2, en este trabajo se propone el uso de técnicas avanzadas de superresolución basadas en modelos de aprendizaje profundo. La superresolución se ha convertido en una herramienta poderosa para mejorar la calidad de las imágenes, utilizando arquitecturas como las redes neuronales convolucionales (CNNs) y las Generative Adversarial Networks (GANs), que permiten generar imágenes de alta resolución mientras se preserva la coherencia espectral y espacial. Estas técnicas, aplicadas a las imágenes multiespectrales de Sentinel-2, tienen el potencial de generar resultados comparables a los obtenidos con sensores de mayor resolución, como el NAIP, pero con la ventaja de trabajar con datos más accesibles.

En diversos estudios, se ha seguido el Protocolo Wald, un estándar en el campo de la teledetección, para validar la calidad de las imágenes generadas mediante superresolución. Este protocolo establece criterios rigurosos que garantizan que las imágenes mejoradas mantengan consistencia espectral y espacial en relación con las imágenes originales. En este trabajo, se propone un modelo de superresolución (SR) que transforma las bandas de 10m y 20m de Sentinel-2 en imágenes de mayor resolución (2.5m), preservando los detalles espectrales críticos para estudios precisos. El proceso de mejora incluye un paso intermedio de superresolución de 10m a 40m, lo que fortalece la robustez del modelo.

Además, el uso del Protocolo Wald asegura que las imágenes fusionadas conserven tanto la integridad espacial como la espectral, validando la calidad de las mismas. Este enfoque, que separa los procesos de fusión de imágenes y mejora de resolución mediante superresolución, permite que cada etapa trabaje de forma independiente, aumentando la versatilidad y aplicabilidad del método propuesto para diferentes tipos de sensores y aplicaciones.

Esta investigación no solo pretende avanzar en el campo de la superresolución de imágenes, sino también ofrecer una solución viable y accesible para aquellos proyectos que requieren imágenes de alta resolución, pero que deben operar dentro de las limitaciones económicas y técnicas de los sensores actuales.


\section{Structure of the Work}


En \textbf{el primer capítulo}, se establecen el propósito y la motivación del trabajo. El problema abordado se centra en la necesidad de mejorar la resolución espacial de las imágenes satelitales, específicamente las del Sentinel-2, mediante técnicas de superresolución. El capítulo también describe la organización de la tesis, delineando las principales secciones y objetivos.

\textbf{El segundo capítulo} profundiza en un análisis detallado del contexto, examinando el estado actual de las tecnologías de imágenes satelitales y la demanda de datos de alta resolución en diversos campos, como la monitorización ambiental y la planificación urbana. Se discuten los beneficios y desafíos de mejorar la resolución espacial mediante la fusión de datos multiespectrales y modelos de superresolución, con un enfoque en las limitaciones de las técnicas existentes.

\textbf{El tercer capítulo} revisa el estado del arte en tecnologías de superresolución aplicadas a imágenes satelitales. El capítulo incluye una visión general de los métodos clave como el \textit{pansharpening}, la inversión de modelos de imagen y los enfoques de aprendizaje profundo. También se destaca la importancia del \textit{Protocolo de Wald} como un marco de validación para garantizar la consistencia espectral y espacial en las imágenes de alta resolución generadas.

En \textbf{el cuarto capítulo}, se definen los objetivos generales y específicos del proyecto. El objetivo principal es desarrollar un modelo de superresolución capaz de mejorar las imágenes del Sentinel-2 de 10m y 20m de resolución a una resolución de 2.5m, utilizando una combinación de modelos basados en redes neuronales convolucionales (CNN) y técnicas de fusión de datos multiespectrales. La sección de metodología detalla el proceso seguido para el entrenamiento del modelo, la preparación del conjunto de datos y la evaluación utilizando el \textit{Protocolo de Wald}.

El \textbf{quinto capítulo} presenta los requisitos funcionales y no funcionales del sistema de superresolución. Los requisitos funcionales incluyen la capacidad de manejar imágenes multiespectrales de diversas resoluciones y la generación de resultados consistentes y de alta calidad con una resolución de 2.5m. Los requisitos no funcionales se centran en la eficiencia, escalabilidad y robustez del sistema al manejar grandes conjuntos de datos.

En \textbf{el sexto capítulo}, se explica en detalle el desarrollo del modelo de superresolución. Esto incluye el proceso de entrenamiento, el diseño de las arquitecturas de redes neuronales (Modelo de Fusión X2 y X4) y la integración de técnicas de aprendizaje profundo. Las limitaciones encontradas durante la fase de entrenamiento, como las restricciones de memoria y el rendimiento del modelo, se discuten junto con posibles optimizaciones.

El \textbf{séptimo capítulo} evalúa el rendimiento del modelo desarrollado, comparándolo con técnicas existentes. La evaluación se basa en métricas tanto cualitativas como cuantitativas, centrándose en la fidelidad espacial y espectral de las imágenes de alta resolución generadas. El capítulo también incluye un análisis de la usabilidad del sistema y su impacto potencial en aplicaciones del mundo real, como la agricultura de precisión y el monitoreo del uso del suelo.

\textbf{El octavo capítulo} concluye el trabajo, resumiendo las principales contribuciones de la tesis y proponiendo futuras direcciones de investigación. Las posibles mejoras incluyen el perfeccionamiento de la arquitectura de aprendizaje profundo, la exploración de otras técnicas de fusión y la aplicación del modelo a diferentes tipos de imágenes satelitales.

\textbf{El Apéndice I} incluye imágenes de salida generadas por el modelo de superresolución, demostrando la resolución mejorada para varias bandas del Sentinel-2.

\textbf{El Apéndice II} presenta los resultados de las pruebas de validación del \textit{Protocolo de Wald}, mostrando el rendimiento del sistema en el mantenimiento de la consistencia espectral y espacial.

\textbf{El Apéndice III} proporciona documentación detallada de las herramientas de software y bibliotecas utilizadas para la implementación del modelo.

\textbf{El Apéndice IV} contiene el artículo de investigación asociado con esta tesis, enviado para su publicación.

