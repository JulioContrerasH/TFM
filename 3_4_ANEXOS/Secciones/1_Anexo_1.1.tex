% \subsection*{Respuesta Sísmica de una Edificación con AS}
%\phantomsubsection
% \addcontentsline{toc}{subsection}{Respuesta Sísmica de una Edificación con AS}



\addcontentsline{toc}{section}{Operacionalización de variables}

\vspace*{2mm}

\begin{table}[H]
    \caption{\doublespacing \\ \textit{Operacionalización de variables}}
    \begin{spacing}{8}
        \fontsize{8pt}{2pt}\selectfont  
        \begin{tabularx}{\linewidth}{P{2.5cm}P{2.6cm}P{4cm}P{2.5cm}P{2.5cm}} % *{4}{P{3cm}}
            \toprule
            % \multicolumn{1}{c}{\textbf{Variables}} & \multicolumn{1}{c}{\textbf{Subvariables}} & \multicolumn{1}{c}{\textbf{Operacionalización}} & \multicolumn{1}{c}{\textbf{Unidad de medida}} & \multicolumn{1}{c}{\textbf{Instrumento}} \\
            \multicolumn{1}{c}{\textbf{Variables}} & \multicolumn{1}{c}{\textbf{Dimensiones}} & \multicolumn{1}{c}{\textbf{Indicadores}} & \multicolumn{1}{c}{\textbf{Unidad de medida}} & \multicolumn{1}{c}{\textbf{Instrumento}} \\
            \midrule
            Inteligencia artificial (independiente) & Corrección geométrica & Error RMS después del ajuste geométrico & Píxeles (px) & Python (LightGlue) \\
            \addlinespace
            & Alineación espectral & Coeficiente de correlación entre las imágenes MSS y TM armonizadas espacialmente & Coeficiente de correlación (r) & Python (Pytorch) \\
            \addlinespace
            & Generación de bandas faltantes & Número de bandas generadas para completar MSS comparable con TM & Número de bandas (nb) & Python (Pytorch) \\
            \addlinespace
            \addlinespace
            Armonización de imágenes satelitales (dependiente) & Precisión de alineación & Precisión de la superposición de píxeles en imágenes armonizadas & Metros (m) & Python (GDAL, Rasterio) \\
            \addlinespace
            & Similitud espectral & Índice de similitud espectral entre imágenes MSS y TM & Sin unidades & Python (PyTorch) \\
            \addlinespace
            & Resolución espacial & Resolución espacial de las imágenes armonizadas & Metros por píxel (m/px) & Python (Rasterio) \\
            \addlinespace
            & Integridad de datos temporales & Cobertura temporal completa en el cubo de datos armonizado & Porcentaje (\%) & Python (xarray) \\
            \bottomrule
        \end{tabularx}
    \end{spacing}
    \vspace{1\baselineskip}
    % \textit{Nota.} Esta tabla muestra las variables operacionalizadas, destacando cómo la inteligencia artificial contribuye a la armonización de imágenes satelitales con modelos de aprendizaje profundo reflejados en las subvariables y métricas, utilizando Python como instrumento clave de implementación.
    \textit{Nota.} Esta tabla muestra las variables operacionalizadas, destacando cómo la inteligencia artificial contribuye a la armonización de imágenes satelitales con modelos de aprendizaje profundo reflejados en las dimensiones y métricas, utilizando Python como instrumento clave de implementación.
    \label{UsoLandsat1}
\end{table}




\addcontentsline{toc}{section}{Matriz de consistencia}

\vspace*{2mm}

\begin{table}[H]
    \caption{\doublespacing \\ \textit{Matriz de consistencia.}}
    \centering
    \begin{spacing}{8}
        \fontsize{8pt}{2pt}\selectfont
        \begin{tabularx}{\textwidth}{@{}XXX@{}}
            \toprule
            \multicolumn{1}{c}{\textbf{Pregunta general}} & \multicolumn{1}{c}{\textbf{Objetivo general}} & \multicolumn{1}{c}{\textbf{Hipótesis general}} \\
            \midrule
            ¿Cómo armonizar las imágenes Landsat MSS para su uso en el monitoreo global y a largo plazo, utilizando inteligencia artificial? & Armonizar las imágenes Landsat MSS utilizando inteligencia artificial para su uso el monitoreo global y a largo plazo & El uso de inteligencia artificial para armonizar las imágenes Landsat MSS permitirá que adquieran propiedades de las imágenes TM, haciendo viable su uso en monitoreos globales de largo plazo \\
            \addlinespace
            \midrule
            \textbf{Preguntas específicas} & \textbf{Objetivos específicos} & \textbf{Hipótesis específicas} \\
            \midrule
            ¿Cómo integrar el aprendizaje profundo y procesamiento de imágenes en la corrección geométrica de las imágenes Landsat MSS y alinearlas con las TM a nivel de pixel y subpixel? & Integrar técnicas de aprendizaje profundo y procesamiento de imágenes para la corrección geométrica de las imágenes Landsat MSS, buscando una alineación precisa con las imágenes TM a niveles de pixel y sub-pixel & La integración de aprendizaje profundo y procesamiento de imágenes mejorará la corrección geométrica de las imágenes Landsat MSS, facilitando su armonización con las imágenes TM \\
            \addlinespace
            ¿De qué manera puede el modelo de aprendizaje profundo MSS2TM alinear espectral y espacialmente las imágenes Landsat MSS y TM? & Desarrollar el modelo de aprendizaje profundo MSS2TM para alinear espectral y espacialmente las imágenes Landsat MSS y TM & El modelo MSS2TM, basado en aprendizaje profundo, logrará una alineación precisa tanto espectral como espacial entre las imágenes Landsat MSS y TM \\
            \addlinespace
            ¿Mediante qué técnica de aprendizaje profundo se puede generar bandas faltantes en imágenes MSS que existen en las TM? & Implementar una técnica de aprendizaje profundo para generar bandas ausentes en imágenes MSS que existen en las TM & La técnica de aprendizaje profundo propuesta permitirá completar las bandas ausentes en las imágenes Landsat MSS, logrando una similitud significativa con las bandas presentes en las TM \\
            \bottomrule
        \end{tabularx}
    \end{spacing}
    \label{MatrizConsistencia}
\end{table}




% \begin{table}[H]
%     \caption{\doublespacing \\ \textit{Comparación y visualización de bandas y longitudes de onda de sensores Landsat mediante Spectral Viewer del U.S. Geological Survey.}}
%     \begin{spacing}{8}
%         \fontsize{8pt}{2pt}\selectfont  
%         \begin{tabularx}{\linewidth}{P{3cm}*{10}{c}} 
%             \toprule
%             \textbf{Designaciones de banda} & \multicolumn{2}{c}{\textbf{L8-9 OLI/TIRS}} & \multicolumn{2}{c}{\textbf{L7 ETM+}} & \multicolumn{2}{c}{\textbf{L4-5 TM}} & \multicolumn{2}{c}{\textbf{L4-5 MSS*}} & \multicolumn{2}{c}{\textbf{L1-3 MSS*}} \\
%             \midrule
%             & B & Longitud & B & Longitud & B & Longitud & B & Longitud & B & Longitud \\
%             \midrule
%             Costera/Aerosol & 1 & 0.43–0.45 & -- & -- & -- & -- & -- & -- & -- & -- \\
%             Azul & 2 & 0.45–0.51 & 1 & 0.45–0.52 & 1 & 0.45–0.52 & -- & -- & -- & -- \\
%             Verde & 3 & 0.53–0.59 & 2 & 0.52–0.60 & 2 & 0.52–0.60 & 1 & 0.5–0.6 & 4 & 0.5–0.6 \\
%             Pancromática** & 8  & 0.50–0.68 & 8 & 0.52–0.90 & -- & -- & -- & -- & -- & -- \\
%             Rojo & 4 & 0.64–0.67 & 3 & 0.63–0.69 & 3 & 0.63–0.69 & 2 & 0.6–0.7 & 5 & 0.6–0.7 \\
%             Infrarrojo cercano & 5 & 0.85–0.88 & 4 & 0.77–0.90 & 4 & 0.76–0.90 & 3 & 0.7–0.8 & 6 & 0.7–0.8 \\
%             Infrarrojo cercano & -- & -- & -- & -- & -- & -- & 4 & 0.8–1.1 & 7 & 0.8–1.1 \\
%             Cirrus & 9 & 1.36–1.38 & -- & -- & -- & -- & -- & -- & -- & -- \\
%             Infrarrojo corto-1 & 6 & 1.57–1.65 & 5 & 1.55–1.75 & 5 & 1.55–1.75 & -- & -- & -- & -- \\
%             Infrarrojo corto-2 & 7 & 2.11–2.29 & 7 & 2.09–2.35 & 7 & 2.08–2.35 & -- & -- & -- & -- \\
%             Térmico & 10 T1 & 10.60–11.19 & 6 T2 & 10.40–12.50 & 6 T2 & 10.40–12.50 & -- & -- & -- & -- \\
%             Térmico & 11 T1 & 11.50–12.51 & -- & -- & -- & -- & -- & -- & -- & -- \\
%             \bottomrule
%         \end{tabularx}
%     \end{spacing}
%     \vspace{1\baselineskip}
%     \textit{Nota.} Hay observaciones que se debe tener en cuenta. Adapatado de \textcite{Landsat2023}. \\
%         * Adquirido a 80 metros, remuestreado a 60 metros. \\
%         ** 15 metros (pancromático). \\
%         T1 = Térmico (adquirido a 100 metros, remuestreado a 30 metros). \\
%         T2 = Térmico (adquirido a 120 metros, remuestreado a 30 metros). 
%     \label{BandasLandsat}
% \end{table}